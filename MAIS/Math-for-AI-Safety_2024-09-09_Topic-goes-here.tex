\documentclass[11pt]{article}
\usepackage{amsmath,verbatim,amssymb,titletoc,enumerate}
\usepackage{bbm}
\usepackage{hyperref}
\hypersetup{colorlinks}

\setlength{\evensidemargin}{.25in}
\setlength{\textwidth}{6in}
\setlength{\topmargin}{-0.4in}
\setlength{\textheight}{8.5in}


\newcommand{\handout}[5]{
   \renewcommand{\thepage}{\href{/MAIS}{Math for AI Safety}: #1 $\cdot$ \arabic{page}}
   \noindent
   \begin{center}
   \framebox{
      \vbox{
    \hbox to 5.78in {{\sf \href{/MAIS/}{Math for AI Safety}} 
\hfill \sf #2 }
       \vspace{4mm}
       \hbox to 5.78in { \sf {\Large \hfill #5  \hfill} }
       \vspace{2mm}
       \hbox to 5.78in { \sf {\em #3 \hfill #4} }
      }
   }
   \end{center}
   \vspace*{4mm}
}

\newcommand{\lecture}[4]{\handout{#1}{#2}{#3}{Notes by #4}{#1}}


\textwidth=6in
\oddsidemargin=0.25in
\evensidemargin=0.25in
\topmargin=-0.1in
\footskip=0.8in
\parindent=0.0cm
\parskip=0.3cm
\textheight=8.00in
\setcounter{tocdepth} {3}
\setcounter{secnumdepth} {2}
\sloppy

\newtheorem{theorem}{Theorem}
\newtheorem{lemma}[theorem]{Lemma}
\newtheorem{proposition}[theorem]{Proposition}
\newtheorem{corollary}[theorem]{Corollary}
\newtheorem{fact}[theorem]{Fact}
\newtheorem{definition}[theorem]{Definition}
\newtheorem{remark}[theorem]{Remark}
\newtheorem{conjecture}[theorem]{Conjecture}
\newtheorem{question}[theorem]{Question}
\newtheorem{answer}[theorem]{Answer}
\newtheorem{exercise}[theorem]{Exercise}
\newtheorem{example}[theorem]{Example}
\newenvironment{proof}{\noindent \textbf{Proof:}}{$\Box$}

\newcommand{\indep}{\perp \!\!\! \perp} % independence
\newcommand{\given}{\,|\,} % notation for conditional independence: $X \indep Y \given Z$ 

\newcommand{\N}{\mathbb N} % natural numbers 0,1,2,...
\newcommand{\Z}{\mathbb Z}  % integers
\newcommand{\R}{\mathbb R} % reals
\newcommand{\C}{\mathbb C} % complex numbers
\newcommand{\F}{\mathbb F} % finite fields

\newcommand{\floor}[1]{\left\lfloor {#1} \right\rfloor} % floor function
\newcommand{\ceiling}[1]{\left\lceil {#1} \right\rceil} % ceiling function
\newcommand{\binomial}[2]{\left( \begin{array}{c} {#1} \\ 
                        {#2} \end{array} \right)} % binomial coefficients
\newcommand{\modulo}[1]{\quad (\mbox{mod }{#1})} %congruences

\newcommand{\ignore}[1]{} % useful for commenting things out
\newcommand{\blue}[1]{\textcolor{blue}{#1}} 
\newcommand{\red}[1]{\textcolor{red}{#1}}
\newcommand{\define}[1]{\emph{\textbf{#1}}}


\begin{document}
\lecture{Lecture Topic Goes Here} % Replace by the topic of the class you're taking notes for!
{\href{/}{Lionel Levine}} % Instructor's name
{September 9, 2024} % Date of the class you're taking notes for
{Your Name Here} % Notetaker's name


\section{Brownian motion appears everywhere}

Brownian motion appears everywhere, even in this template:

\begin{theorem}
Let $B$ be a standard Brownian motion, and let $a >0$. Then
	\[ (B_t)_{t \geq 0} \stackrel{d}{=} (a^{-1/2} B_{at})_{t \geq 0} \]
\end{theorem}

\begin{proof}
They're both mean zero Gaussian processes with continuous sample paths and covariance $E[B_s B_t] = \min(s,t)$.
\end{proof}

\red{Wait, why does that prove it?}

\begin{lemma}
The law of a Gaussian process with continuous sample paths is determined by its means and covariances.
\end{lemma}

\begin{example}
If $X,Y$ are independent $N(0,1)$ then so are $\frac{X+Y}{\sqrt{2}}$, $\frac{X-Y}{\sqrt{2}}$.
\end{example}

\blue{Remind me what a Gaussian process is?}

\begin{definition}
Let $I$ be an arbitrary index set. A stochastic process $(X_i)_{i \in I}$ is a \define{Gaussian process} if every finite linear combination  
	\[ a_1 X_{i_1} + \ldots + a_n X_{i_n} \] 
has a normal distribution for all $a_1, \ldots, a_n \in \R$ and all $i_1, \ldots, i_n \in I$.
\end{definition}

\section{Insert your notes here!}

\ignore{This sentence won't appear in the pdf output.} % neither will anything preceded by a percent sign

% Now remove the stuff about Brownian motion, and replace it with your notes.  Go to it, notetaker!

\end{document}
