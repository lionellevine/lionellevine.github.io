\documentclass{article}
\usepackage{amsmath}
\usepackage{amssymb}
\usepackage{amsthm}
\usepackage{enumerate}
\usepackage{graphicx}
\begin{document}
\begin{center}{\large David Thomas, 18.312 Lecture 3} \end{center}\\

\begin{flushleft}
Announcements:
\end{flushleft}
Office hours changed to Tuesday 12-1pm and Wednesday 1-2pm.

\begin{flushleft}
Today:
\end{flushleft}
\begin{enumerate} \item[(1)] M$\"{o}$bius Inversion Example
\item[(2)] Multiplicative Functions, Dirichlet Series
\item[(3)] Permutations, Stirling Numbers

\begin{center}\framebox{\mboxbf{M$\"{o}$bius Inversion Example}} \end{center}
\\
Recall M$\"{o}$bius Inversion from last lecture. Let $f,g$ be functions on $\mathbb{N}$, then $f(n) = \sum_{d|n}g(d)$ if and only if $g(n) = \sum_{d|n}\mu(d)f(\frac{n}{d})$ where $\mu(d)$ is the Mobius inversion function.\\
\\
\\
\noindent Fix $n\in \mathbb{N}$ and let 
$$P(d) = \#\{\mbox{Primitive necklaces }(a_1,\dots,a_d)|a_i \in [n] \}$$
 Primitive means that all rotations are distinct ie. $r^i(\underline{a}) \not= r^j(\underline{a})$, for $i\not=j$ and $i,j \in [d]$. Fix $k\in \mathbb{N}$, let 
\begin{align*} 
N(k) &= \#\{\mbox{all necklaces }(a_1,\ldots,a_k)|a_i\in [n]\}\\
&= n^k\\ 
&= \sum_{d|k}P(d)
\end{align*}
Now using M$\"{o}$bius inversion we have 
\begin{align*} P(k) &= \sum_{d|k} \mu(d)N(\frac{k}{d})\\
&= \sum_{d|k}\mu(d)n^{k/d}\end{align*}
 where $P(k)$ is divisible by $k$. Now we can reduce this to Fermats little theorem by setting $k=p$, then 
 \begin{align*}P(p) &= \mu(1)n^p +\mu(p)n\\
 &= n^p-n\end{align*}
\\
\noindent Now fix $n\in\mathbb{N}$, let $$M(k) = \#\{\mbox{equivalence classes of necklaces }(a_1,\dots,a_k), a_i \in [n], \mbox{up to rotations}\}$$
For example if $n=2, k=4$ then
$$M(4) = 6$$
\begin{center}
\includegraphics[scale = 0.35]{m4.png}
\end{center}
\\
\noindent Any necklace $\underline{a}$ has a stabilizer $stab(a) \subseteq C_k = <r>$ and $stab(a) = C_d = <r^{k/d}>$ for some divisor $d|k$. I will show that $$ \# \{\underline{a}|stab(\underline{a}) = C_d\} = \frac{P(k/d)}{k/d}$$
\begin{center}
\includegraphics[scale = 0.30]{dia1.png}
\end{center}
\\
Since the stabilizer of $\underline{a}$ is $C_d$ we have blocks of length $k/d$ of the necklace that are repeated d times. $P(k/d)$ counts the different kinds of blocks. We must divide by $k/d$ to correct for overcounting blocks that are rotations of each other, and hence are the same. Combining these into $\frac{P(k/d)}{k/d}$ counts the number of unique necklaces up to rotation with stabilizer $C_d$.

\\
Hence \begin{align*} M(k) &= \sum_{d|k} \frac{P(k/d)}{k/d}\\
&= \sum_{d|k} \frac{P(d)}{d}\\
& \mbox{ and substituting P(d) from earlier result yields}\\
M(k) &= \sum_{d|k}\frac{1}{d} \sum_{l|d}\mu(l)n^{d/l}\\
&= \sum_{d|k}\frac{1}{d}  \sum_{l|d}\mu(\frac{d}{l})n^l\\
&= \sum_{l|k} n^l \sum_{l|d|k}\frac{1}{d}\mu(\frac{d}{l})\\
& \mbox{letting $m = \frac{d}{l}$, we have}\\
M(k) &= \sum_{l|k} n^l \sum_{m|\frac{k}{l}} \frac{\mu(m)}{ml}\\
&= \sum_{l|k} \frac{n^l}{l} \sum_{m|\frac{k}{l}}\frac{\mu(m)}{m}\\
& \mbox{and }\sum_{m|\frac{k}{l}}\frac{\mu(m)}{m} = \frac{\phi (k/l)}{k/l}\\
M(k) &= \sum_{l|k} \frac{n^l}{l} \frac{\phi(k/l)}{k/l}\\
&= \frac{1}{k} \sum_{l|k} \phi(k/l)n^l
\end{align*}

Now for a quick review of Burnside's Lemma before we show how to use it for a more concise proof of our result. \\

\newtheorem*{lem}{Lemma}
\begin{lem} Group Action $G\times X \to X$, where $G$ and $X$ are finite. The number of orbits equals $\frac{1}{|G|}\sum_{g\in G} \psi(g)$, where $\psi(g) = \#\{x\in X|gx=x\}$. \end{lem}

\begin{proof}
\begin{align*}
\sum_{g\in G} \psi(g) &= \#\{(g,x) \in G \times X | gx =x\}\\
&= \sum_{x\in X} \#\{g\in G| gx=x\}\\
&= \sum_{x\in X}|stab(x)|\\
&= \sum_{x\in X} \frac{|G|}{|Orb(x)|}\\
&= |G|(\# \mbox{ of orbits })
\end{align*}
\end{proof}

Now let's apply to our case: $G = C_k = \{1,r^1,r^2,\dots, r^{k-1}\}, X = \{\mbox{ all necklaces }(a_1,\dots, a_k)| a_i \in [n]\}$, and $\psi(r^i) = n^d$ where $d = GCD(i,k)$. \\
\\
{\small \noindent *note: $d = GCD(i,k)$, then $r^i(\underline{a}) = (\underline{a})$ if and only if $r^d(\underline{a}) = (\underline{a})$ and $\#\{ i \in [k]|GCD(i,k) = d\} = \phi(k/d)$}
\\
\\
\noindent Then Burnside's Lemma gives us
$$ M(k) = \frac{1}{k} \sum_{i=1}^k \psi(r^i) = \frac{1}{k}\sum_{d|k}\phi(k/d)n^d$$ 


\begin{center}\framebox{\mboxbf{Multiplicative Functions, Dirichlet Series}} \end{center}
\\
Let $f,g:\mathbb{N} \to \mathbb{C}$. We will denote convolution by *. Then $(f*g)(n) = \sum_{d|n}f(d)g(\frac{n}{d})$. It is useful to consider dirichlet series which are functions of the form $F(s) = \sum _{n \geq 1}\frac{f(n)}{n^s}$. The product of two dirichlet functions is 
\begin{align*} F(S)G(S) &= (\sum _{n \geq 1}\frac{f(n)}{n^s})(\sum _{n \geq 1}\frac{g(n)}{n^s})\\
&= \sum_{k \geq 1}\sum_{l \geq 1} \frac{f(k)g(l)}{(kl)^s}\\
&= \sum_{n \geq 1} \frac{\sum_{kd = n}f(k)g(l)}{n^s}\\
&= \sum_{n\geq 1}\frac{(f*g)(n)}{n^s}
\end{align*}
which is also a dirichlet series/function.\\
\\
\noindent \mboxbf{Definition:} A function f is \emph{multiplicative} if f(mn) = f(m)f(n) whenever GCD(m,n) =1.\\
\\
For example $\phi, \mu, n^\alpha, \tau(n) = \#\{\mbox{divisors of n}\}, \sigma_\alpha (n) = \sum_{d|n} d^\alpha$ are all multiplicative functions. \\
\\
\noindent If f is multiplicative, then 
$\sum_{n \geq 1} \frac{f(n)}{n^s} = \prod_{\mbox{p prime}} (f(1)+\frac{f(p)}{p^s} + \frac{f(p^2)}{p^{2s}}+\dots)$, which is called the Euler Product.\\
\\
For example, let $f(n)=1$ forall $n$, and let 
\begin{align*}\zeta(s) &= \sum_{n\geq 1} \frac{1}{n^s}\\
&= \prod_{\mbox{p prime}}(1+ \frac{1}{p^s}+ \frac{1}{p^{2s}} + \dots) \\
&= \prod_{\mbox{p prime}} \frac{1}{1-p^{-s}}\\
\mbox{Then }&\\
\frac{1}{\zeta(s)} &= \prod_{\mbox{p prime}} (1- p^{-s})\\
&= \sum_{n \geq 1} \frac{f(n)}{n^s}\\
\mbox{where $f(p_1\dots p_k) = (-1)^k$ if $p_i$}&\mbox{ distinct primes and $f(n) = 0$ if $n$ is not square-free}\\
\end{align*}
Hence $$\sum_{n \geq 1} \frac{f(n)}{n^s} = \sum_{n \geq 1} \frac{\mu(n)}{n^s}$$
\\
\\
\noindent This gives us more ways to express M$\"{o}$bius Inversion:
$$f(n) = \sum_{d|n}g(d) \leftrightarrow g(n) = \sum_{d|n}\mu(\frac{n}{d})f(d)$$
or equivalently
$$f = g*1 \leftrightarrow g = \mu * f $$
or equivalently
$$F(s)= G(s)\zeta(s) \leftrightarrow G(s) = \frac{1}{\zeta(s)} F(s)$$
\\
\\
To end this section we will explore two more properties of convolution. \\
\\
\mboxbf{Note:} * is associative!
$$(f*g)*h = f*(g*h)$$
$$(FG)H = F(GH)$$

Example: Compute $\sum_{d|n} \phi(d) \tau(\frac{n}{d})$, where $\tau(n) = \#\{d| \mbox{d divides n}\}$.
\begin{align*}
 \sum_{d|n} \phi(d) \tau(\frac{n}{d}) &= ( \phi * \tau)(n) \\
 &= ((\mu *n)*(1*1))(n)\\
 &= ((\mu*1)*(n*1))(n)
\end{align*}

Let $\delta = (\mu*1)$, then $\delta(n) = 1$ when $n =1$ and $0$ for $n\geq 2$. $\delta$ is the identity for convolution, $\delta*f=f$. Hence counting with our problem we have 
\begin{align*}
((\mu*1)*(n*1))(n) &= (n*1)(n)\\
&= \sum_{d|n} d
\end{align*}
\\
\\
\noindent \mboxbf{Note:} if f,g are multiplicative, then f*g is also multiplicative.

\begin{align*}
\sum_{n\geq 1} \frac{(f*g)(n)}{n^s} &= (\sum_{n\geq 1} \frac{f(n)}{n^s})(\sum_{n\geq 1} \frac{g(n)}{n^s})\\
&= \prod_{\mbox {p prime}}(\sum_{k\geq 0}\frac{f(p^k)}{p^{ks}})(\sum_{l\geq 0}\frac{g(p^l)}{p^{ls}})\\
&= \prod_{\mbox {p prime}} \sum_{k,l \geq 0} \frac{f(p^k)g(p^l)}{p^{ks}p^{ls}}\\
&= \prod_{\mbox {p prime}} \sum_{m \geq 0} \frac{1}{p^{ms}} \sum_{k+l = m} f(p^k)g(p^l)\\
&= \prod_{\mbox {p prime}} \sum_{m \geq 0} \frac{1}{p^{ms}} \sum_{d|p^m} f(d) g(\frac{p^m}{d})\\
&= f*g(p^m) 
\end{align*}

\begin{center}\framebox{\mboxbf{Permutations and Stirling Numbers}} \end{center}
\\

Permutation $\pi \in S_n$ where $S_n = \{\mbox{bijections from }[n] \to [n] \}$

In two-line notation \\
\begin{center}
\begin{verbatim}
1 2 3 4 5 6

--------------

4 1 5 3 2 6
\end{verbatim}
\end{center}
means $\pi(1)=4,\pi(2) = 1,\pi(3)=5,\pi(4)=3,\pi(5)=2,\pi(6)=6$. In cycle notation this permutation $\pi$ would be represented by (14352)(6) as shown below
\begin{center}
\includegraphics[scale = 0.5]{myimage.png}
\end{center}

Let $c(n,k)$ (signless Stirling number of the first kind) be the number of $\pi \in S_n$ that have exactly k cycles. For example, 
$$c(n,n) =1$$ 
This is because there is only one way to put each element in [n] into its own cycle.
$$c(n,1) = (n-1)!$$
Letting 1 be the first element we list in the cycle notation (ie. (1 $\dots$ )), then there are (n-1)! different ways to order the elements that come next which correspond to the different ways to arrange the cycle. 

$$ c(n,n-1) = \begin{pmatrix} n \\ 2 \end{pmatrix}$$
Let $c_i$ denote the length of the ith cycle and $c_i \leq c_{i+1}$. Then $\sum_{i \in [n-1]}c_i = n$ and $c_i \geq 1$. It follows that $c_i = 1$ for $i \in [n-2]$ and $c_{n-1} = 2$. Hence there are n-2 cycles of length 1 and one cycle of length 2. There are $\begin{pmatrix} n \\ 2 \end{pmatrix}$ ways to choose the elements that are in the 2-cycle. Since cycle $(ab)= (ba)$, $\begin{pmatrix} n \\ 2 \end{pmatrix}$ is not undercounting and the result follows.

\begin{lem} $c(n,k) = (n-1)c(n-1,k)+c(n-1,k-1)$ \end{lem}
\begin{proof}
Given a permutation $\pi \in S_{n-1}$, one can either
\begin{enumerate}
\item[(1)] Insert n in an existing cycle, $(n-1)c(n-1,k)$
\item[(2)] Make n into its own cycle, $c(n-1,k-1)$
\end{enumerate}
\end{proof}

Here are a few values for $c(n,k)$:
\begin{tabular}{cc|cccc}
&&&k&&\\
&0&1&2&3&4\\
\hline
&1&1&0&0&0\\
n&2&1&1&0&0\\
&3&2&3&1&0\\
&4&6&11&6&1\\
\end{tabular}
\\
\begin{lem} $\sum^n_{k=1}c(n,k)x^k = x(x+1)(x+2) \dots (x+n-1)$ \end{lem}\\
\\
We will prove this next lecture. 

\end{document}