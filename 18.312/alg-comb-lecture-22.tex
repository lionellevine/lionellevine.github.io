%% LyX 1.6.8 created this file.  For more info, see http://www.lyx.org/.
\documentclass[11pt,english]{article}
\usepackage[T1]{fontenc}
\usepackage[latin9]{inputenc}
\usepackage{float}
\usepackage{amstext}
\usepackage{graphicx}
\usepackage{amssymb}

\makeatletter

%%%%%%%%%%%%%%%%%%%%%%%%%%%%%% LyX specific LaTeX commands.
%% Because html converters don't know tabularnewline
\providecommand{\tabularnewline}{\\}

%%%%%%%%%%%%%%%%%%%%%%%%%%%%%% User specified LaTeX commands.
\usepackage{amssymb}
\usepackage{amsfonts}
\usepackage{amsmath}
\usepackage{bm}
\usepackage{latexsym}
\usepackage{epsfig}
\usepackage{bbm}

\setlength{\evensidemargin}{.25in}
\setlength{\textwidth}{6in}
\setlength{\topmargin}{-0.4in}
\setlength{\textheight}{8.5in}

%       Misc 
\def\todo{\ttfamily\bfseries[TODO:] }
\newcommand{\Mob}[0]{M\"{o}bius }

\newcommand{\ind}{1\hspace{-2.5mm}{1}} 
\newcommand{\handout}[5]{
   \renewcommand{\thepage}{#1-\arabic{page}}
   \noindent
   \begin{center}
   \framebox{
      \vbox{
    \hbox to 5.78in {{\sf 18.312: Algebraic Combinatorics} 
\hfill \sf #2 }
       \vspace{4mm}
       \hbox to 5.78in { {\Large \hfill #5  \hfill} }
       \vspace{2mm}
       \hbox to 5.78in { {\em #3 \hfill #4} }
      }
   }
   \end{center}
   \vspace*{4mm}
}

\newcommand{\lecture}[4]{\handout{#1}{#2}{Lecture date: #3}{Notes by: #4}{Lecture #1}}

\textwidth=6in
\oddsidemargin=0.25in
\evensidemargin=0.25in
\topmargin=-0.1in
\footskip=0.8in
\parindent=0.0cm
\parskip=0.3cm
\textheight=8.00in
\setcounter{tocdepth} {3}
\setcounter{secnumdepth} {2}
\sloppy

\newtheorem{theorem}{Theorem}
\newtheorem{lemma}[theorem]{Lemma}
\newtheorem{proposition}[theorem]{Proposition}
\newtheorem{corollary}[theorem]{Corollary}
\newtheorem{fact}[theorem]{Fact}
\newtheorem{definition}[theorem]{Definition}
\newtheorem{remark}[theorem]{Remark}
\newtheorem{conjecture}[theorem]{Conjecture}
\newtheorem{question}[theorem]{Question}
\newtheorem{answer}[theorem]{Answer}
\newtheorem{exercise}[theorem]{Exercise}
\newtheorem{example}[theorem]{Example}
\newenvironment{proof}{\noindent \textbf{Proof:}}{$\Box$}

\newcommand{\N}{\mathbb N} % natural numbers 0,1,2,...
\newcommand{\Z}{\mathbb Z}  % integers
\newcommand{\R}{\mathbb R} % reals
\newcommand{\C}{\mathbb C} % complex numbers
\newcommand{\F}{\mathbb F} % finite fields

\newcommand{\floor}[1]{\left\lfloor {#1} \right\rfloor} % floor function
\newcommand{\ceiling}[1]{\left\lceil {#1} \right\rceil} % ceiling function
\newcommand{\binomial}[2]{\left( \begin{array}{c} {#1} \\ 
                        {#2} \end{array} \right)} % binomial coefficients
\newcommand{\modulo}[1]{ (\mbox{mod }{#1})} %congruences
%\newcommand{\modulo}[1]{\quad (\mbox{mod }{#1})} %congruences

\newcommand{\ignore}[1]{} % useful for commenting things out

\makeatother

\usepackage{babel}

\begin{document}
\lecture{22}{Lionel Levine}{May 3, 2011}{Lou Odette}


\section*{This lecture:}
\begin{itemize}
\item Smith normal form of an integer matrix (linear algebra over $\mathbb{Z}$).
\end{itemize}

\section{Review of Abelian Groups (= $\mathbb{Z}$-modules)}

Recall that given a ring $R$ with $1$, an $R$-module is an Abelian group
written additively with a map $R\times M\rightarrow M$ ({}``scalar
multiplication'') with $R$ the scalars and $M$ the vectors, satisfying\begin{eqnarray*}
r\left(m_{1}+m_{2}\right) & = & rm_{1}+rm_{2}\\
r\left(sm\right) & = & \left(rs\right)m\\
1m & = & m\end{eqnarray*}
which is analogous to a vector space over $R$, with the difference
that we may lack multiplicative inverses in $R$, by contrast with a vector
space over a field.

If $G$ is an group, we have a map $\mathbb{Z}\times G\rightarrow G$\[
\begin{array}{lc}
\left(n,g\right)\mapsto\underbrace{g+\cdots+g}_{n\text{ times}} & \text{if }n>0\\
\left(0,g\right)\mapsto0\\
\left(n,g\right)\mapsto-\left(\underbrace{g+\cdots+g}_{n\text{ times}}\right) & \text{if }n<0\end{array}\]


so we admit scalar multiplication by integers, but not anything else.  In particular, any abelian group has the structure of a $\Z$-module.

Now, say $G$ is an Abelian group, finitely generated from generators
$g_{1},\ldots,g_{n}$.  Then there is a surjective group homomorphism $f:\mathbb{Z}^{n}\rightarrow G$ taking basis elements to the generators \begin{eqnarray*}
f\left(e_{i}\right) & = & g_{i}\\
f\left(\sum_{i\in\left[n\right]}c_{i}e_{i}\right) & = & \sum_{i\in\left[n\right]}c_{i}g_{i}. \end{eqnarray*}


Let $K$ be a kernel of $f$, the subgroup of $\mathbb{Z}^{n}$ s.t.\[
K=\ker f=\left\{ \left.\sum_{i\in\left[n\right]}c_{i}g_{i}\right|\sum_{i\in\left[n\right]}c_{i}g_{i}=0\text{ in }G\right\} \]


\begin{definition}

A group $G$ is \emph{torsion-free} if\[
\forall g\in G,\; g\neq0,\text{ and }\forall n\in\mathbb{Z},\; n\neq0\text{ we have }ng\neq0\]


\end{definition}

\begin{example}

$\mathbb{Z}$ and $\mathbb{Z}^{n}$ are torsion free, but $\mathbb{Z}/n\mathbb{Z}$
is not torsion free, since $ng=0$ for all $g \in \Z / n\Z$.

\end{example}

Note that if $G$ is torsion free then it is infinite or zero since
with $g\in G$ and $g\neq0$, then $g,2g,3g,\ldots$ are distinct,
since otherwise $ig=jg\Rightarrow\left(i-j\right)g=0$.

By the \textbf{F}undamental \textbf{T}heorem of \textbf{F}initely
\textbf{G}enerated \textbf{A}belian \textbf{G}roups (FTFGAG), any finitely generated abelian group $G$ has the form
\begin{equation}
G\simeq\mathbb{Z}^{r}\times\left(\mathbb{Z}/n_{1}\mathbb{Z}\right)\times\cdots\times\left(\mathbb{Z}/n_{k}\mathbb{Z}\right)\label{eq:FTFGAG}\end{equation}
where $r\ge0$ is unique but the $n_{1},\cdots,n_{k}>1$ are not necessarily
unique.

\begin{example}

$\mathbb{Z}_{6}\times\mathbb{Z}_{4}\simeq\mathbb{Z}_{2}\times\mathbb{Z}_{3}\times\mathbb{Z}_{4}\simeq\mathbb{Z}_{2}\times\mathbb{Z}_{12}$,
since $\mathbb{Z}_{m}\times\mathbb{Z}_{n}=\mathbb{Z}_{nm}$, for $m\perp n$
by the Chinese remainder theorem.

\end{example}

There are two ways to get uniqueness:
\begin{enumerate}
\item require that all the $n_{i}$ are prime powers\label{enu:unique_1}
\item or require $n_{1}|n_{2}|n_{3}\cdots|n_{k}$. \label{enu:unique_2}
\end{enumerate}
We consider the second of these today.

\begin{lemma}

Any subgroup $K\subset\mathbb{Z}^{n}$ satisfies $K\simeq\mathbb{Z}^{r}$
for some $r\le n$. 
\end{lemma}

Note that unlike subspaces of a vector space, is it possible to have $r=n$ and $K\neq\mathbb{Z}^{n}$.  For example, $2\mathbb{Z}^{n} \subsetneq \mathbb{Z}^{n}$ while is it still
the case that $2\mathbb{Z}^{n}\simeq\mathbb{Z}^{n}$.  In this sense abelian groups are ``more interesting'' than vector spaces.

Now, since $K\subset\mathbb{Z}^{n}\Rightarrow K\simeq\mathbb{Z}^{r}$
for some $r\le n$, pick a basis $x_{1},\ldots,x_{r}\in K$ so that\[
K=\left\{ \left.\sum_{i\in\left[r\right]}c_{i}x_{i}\right|c_{i}\in\mathbb{Z}\right\} \]
and define\[
L:\mathbb{Z}^{r}\rightarrow\mathbb{Z}^{n};\; e_{i}\mapsto x_{i}\]
so that 
\[
G \simeq \mathbb{Z}^{n}/K=\mathbb{Z}^{n}/\text{Image}\left(L\right)=\mathbb{Z}^{n}/L\mathbb{Z}^{r}\]
We can think of $L$ as an $r\times n$ matrix and each $x_{i}=\sum_{j\in\left[n\right]}a_{i,j}e_{i}$
for $i\in\left[r\right]$, where the $a_{i,j}$ are the matrix entries
of $L$.  We can extend $L$ to $\mathbb{Z}^{n}$, i.e. $L:\mathbb{Z}^{n}\rightarrow\mathbb{Z}^{n}$
by setting $e_{i}\mapsto0$ for $i>r$ (add zero {}``columns''). 

So far we have seen how defining an abelian group $G$ via generators and relations leads to an $n\times n$ matrix $L$ such that $G \simeq \Z^n / L\Z^n$, where $n$ is the number of generators.
The question that Smith normal form address is: given a group in this form, $\Z^n / L \Z^n$, how do we express it in the factored form (\ref{eq:FTFGAG})?

\begin{example}\label{ex:group}

Let\[
L=\left(\begin{array}{cc}
2 & -1\\
1 & 2\end{array}\right)\rightarrow G=\left\langle g_{1},g_{2}\right\rangle /\begin{array}{c}
2g_{1}-g_{2}=0\\
g_{1}+2g_{2}=0\end{array}\]
and\[
G=\mathbb{Z}^{2}/\left(\begin{array}{cc}
2 & 1\\
-1 & 2\end{array}\right)\mathbb{Z}^{2}\]


%
\begin{figure}[H]
\begin{centering}
\includegraphics[scale=0.8]{L22b}
\par\end{centering}

\caption{{\footnotesize This figure illustrates $G = \Z^2 / L\Z^2$, where $L\Z^2$ consists
of the points $\left(2a+b,2b-a\right)$ for $a,b\in\mathbb{Z}$, as
marked by the symbol $\circ$ on the grid. The remaining symbols represent
elements in the respective equivalence classes.  The points enclosed
by the box represent the members of all equivalence classes, illustrating
that $\left|G\right|=5$.  So $G \simeq \Z/5\Z$.}}

\end{figure}


\end{example}

Now, consider the kinds of changes to $L$ that don't change the isomorphism
type of $G$. One approach is to change the generators.

\begin{example}

Write the group $G$ of example (\ref{ex:group}) using different
generators\[
G=\left\langle h_{1},h_{2}\right\rangle ;\; h_{1}=g_{1},\, h_{2}=3g_{1}+g_{2}\]
In general, if $H\in GL_{n}\mathbb{\left(Z\right)}$, where $GL_{n}\mathbb{\left(Z\right)}$
is the set of $n\times n$ matrices $U$ with $\det U=\pm1$, so the
inverse also is an integer matrix, then since $U\mathbb{Z}^{n}=\mathbb{Z}^{n}$\[
G\simeq\mathbb{Z}^{n}/L\mathbb{Z}^{n}=U\mathbb{Z}^{n}/L\mathbb{Z}^{n}\simeq\mathbb{Z}^{n}/U^{-1}L\mathbb{Z}^{n}\]
In addition, if $V\in GL_{n}\left(\mathbb{Z}\right)$, since $V\mathbb{Z}^{n}=\mathbb{Z}^{n}$\begin{eqnarray*}
\mathbb{Z}^{n}/U^{-1}L\mathbb{Z}^{n} & = & \mathbb{Z}^{n}/U^{-1}L\left(V\mathbb{Z}^{n}\right)\\
 & = & \mathbb{Z}^{n}/\left(U^{-1}LV\right)\mathbb{Z}^{n}\end{eqnarray*}


\end{example}

\begin{example}

Write the group $G$ of example (\ref{ex:group}) with different relations\[
\begin{array}{c}
2g_{1}-g_{2}=0\\
g_{1}+2g_{2}=0\end{array}\rightarrow\begin{array}{c}
2g_{1}-g_{2}=0\\
3g_{1}-g_{2}=0\end{array}\]


\end{example}

\begin{definition}

An $n\times n$ integer matrix $S$ is in \emph{Smith Normal Form}
(SNF) if $S$ is a diagonal matrix and uniqueness condition (\ref{enu:unique_2})
is satisfied with the diagonal elements $\left(S\right)_{i,i}\equiv d_{i}$,
i.e. \[
d_{1}|d_{2}|d_{3}\cdots|d_{n};\; d_{i}\ge0,\,\forall i\in\left[n\right]\]
Note that some $d_{i}$ may be zero, since any integer divides zero. 

\end{definition}

\begin{theorem}

An integer matrix $L=\left(a_{i,j}\right)_{i,j\in\left[n\right]}$
can be written as\[
L=USV\]
where $S$ is in SNF and $U,V\in GL_{n}\mathbb{Z}$ (invertible over
the integers). Moreover, the non-zero $d_{i}$ on the diagonal of
$S$ are unique (note, $\gcd\left(\cdot\right)$ is non-negative by
definition):\begin{eqnarray*}
d_{1} & = & \gcd\left(a_{i,j}\right)\\
d_{1}d_{2} & = & \gcd\left(a_{i,j}a_{k,l}-a_{i,l}a_{j,k}\right)\qquad\left(\text{i.e. the }2\times2\text{ determinants}\right)\\
 & \vdots\\
d_{1}\cdots d_{k} & = & \gcd\left(\text{all }k\times k\text{ minors of }L\right)\\
 & \vdots\\
d_{1}\cdots d_{n} & = & \left|\det L\right|\end{eqnarray*}
with $\left|G\right|=\left|\det L\right|$. In terms of the group
$G$, if $L$ has SNF $S$ then \begin{eqnarray*}
G=\mathbb{Z}^{n}/L\mathbb{Z}^{n} & \simeq & \mathbb{Z}^{n}/S\mathbb{Z}^{n}\\
 & \simeq & \left\langle g_{1},\cdots,g_{n}\right\rangle /\left(d_{i}g_{i}=0,\, i\in\left[n\right]\right)\\
 & \simeq & \left\langle g_{1}\right\rangle /\left(d_{1}g_{1}\right)\times\cdots\times\left\langle g_{n}\right\rangle /\left(d_{n}g_{n}\right)\\
 & \simeq & \mathbb{Z}/\left(d_{1}\mathbb{Z}\right)\times\cdots\times\mathbb{Z}/\left(d_{n}\mathbb{Z}\right)\end{eqnarray*}
in particular
\begin{itemize}
\item the rank of $G$ is $\#\left\{ i|d_{i}=0\right\} $
\item if $G$ is finite (all $d_{i}>0$) then $\left|G\right|=d_{1}\cdots d_{n}=\left|\det L\right|$
\end{itemize}
Note: row and column operations don't change the $gcd\left(\cdot\right)$
result since, if $n_{1},\ldots,n_{k}\neq0$ (noting that if $\left(m,n\right)=1$
then $\exists c,c'\in\mathbb{Z}$ s.t. $cm+c'n=1$) \[
\gcd\left(n_{1},\ldots,n_{k}\right)=\min\left\{ \left.d>0\right|d=\sum_{i\in\left[n\right]}c_{i}n_{i}\text{ for some }c_{1},\cdots,c_{k}\in\mathbb{Z}\right\} \]
so $L=USV$.

\end{theorem}

\begin{example}

Consider\[
L=\left(\begin{array}{ccc}
1 & 3 & 1\\
3 & 1 & 3\\
1 & 3 & 5\end{array}\right)\rightarrow\mathbb{Z}^{3}/L\mathbb{Z}^{3}\simeq\mathbb{Z}_{4}\times\mathbb{Z}_{8}\]
since\begin{eqnarray*}
d_{1} & = & \gcd\left(a_{i,j}\right)=1\\
d_{1}d_{2} & = & \gcd\left(-8,0,8,0,4,12,8,12,-4\right)=4\\
d_{1}d_{2}d_{3} & = & \left|\det L\right|=32\end{eqnarray*}


\end{example}


\section{Commutative Monoids}

\begin{definition}

A monoid $M$ is a set with an associative operation\[
\mu:M\times M\rightarrow M\]
and an identity element $e\in M$\[
\left(e,m\right)\mapsto m,\,\forall m\in M\]
$M$ is commutative if $\mu\left(m_{1},m_{2}\right)=\mu\left(m_{2},m_{1}\right)$.

\end{definition}

We are interested in how to get a group from this object

\begin{example}\label{monoid_example}

Consider\begin{eqnarray*}
m & = & \left\langle g\right\rangle /\left(10g=6g\right);\;\left(k+4\right)_{g}=kg,\,\forall g\ge6\\
 & = & \left\{ e,g,2g,\ldots,9g\right\} \end{eqnarray*}
and so, writing $\mu$ as addition \[
8g+8g=16g=12g=8g\]


\end{example}

\begin{definition}

An ideal of a monoid $M$ is a subset satisfying\[
I\subseteq M\text{ s.t. }I+M\subseteq I\]
i.e. $\forall x\in I,\,\forall m\in M$ we have $m+x\in I$.

\end{definition}

\begin{theorem}

Let $M$ be a finite commutative monoid and let $J$ be the minimal
ideal of $M$\[
J=\bigcap_{\text{ideals }I}I\]
Then $J$is an Abelian group.

\end{theorem}

In example (\ref{monoid_example}) \begin{eqnarray*}
I & = & \left\{ 8g\right\} +M\\
 & = & \left\{ \left.8g+m\right|m\in M\right\} \end{eqnarray*}
e.g.\[
\begin{array}{c}
8g+g=9g\\
8g+2g=6g\\
8g+3g=7g\\
8g+4g=8g\end{array}\rightarrow I=\left\{ 6g,7g,8g,9g\right\} \]
and in the table below, the second last column is the identity, while
the last column is cyclic of order $4$, with $9g$ the generator

\begin{center}
\begin{tabular}{|c||c|c|c|c|}
\cline{2-5} 
\multicolumn{1}{c|}{} & $6g$ & $7g$ & $8g$ & $9g$\tabularnewline
\hline
\hline 
$6g$ & $8g$ & $9g$ & $6g$ & $7g$\tabularnewline
\hline 
$7g$ & $9g$ & $6g$ & $7g$ & $8g$\tabularnewline
\hline 
$8g$ & $6g$ & $7g$ & $8g$ & $9g$\tabularnewline
\hline 
$9g$ & $7g$ & $8g$ & $9g$ & $6g$\tabularnewline
\hline
\end{tabular}
\par\end{center}
\end{document}
