%% LyX 1.6.8 created this file.  For more info, see http://www.lyx.org/.

\documentclass[english]{article}
\usepackage[T1]{fontenc}
\usepackage[latin9]{inputenc}
\usepackage{textcomp}
\usepackage{amstext}
\usepackage{graphicx}
\usepackage{amssymb}

\makeatletter

%%%%%%%%%%%%%%%%%%%%%%%%%%%%%% LyX specific LaTeX commands.
\newcommand{\lyxmathsym}[1]{\ifmmode\begingroup\def\b@ld{bold}
  \text{\ifx\math@version\b@ld\bfseries\fi#1}\endgroup\else#1\fi}

%% Because html converters don't know tabularnewline
\providecommand{\tabularnewline}{\\}

%%%%%%%%%%%%%%%%%%%%%%%%%%%%%% User specified LaTeX commands.
\usepackage{amssymb}
\usepackage{amsfonts}
\usepackage{amsmath}
\usepackage{bm}
\usepackage{latexsym}
\usepackage{epsfig}
\usepackage{bbm}

\setlength{\evensidemargin}{.25in}
\setlength{\textwidth}{6in}
\setlength{\topmargin}{-0.4in}
\setlength{\textheight}{8.5in}

\newcommand{\ind}{1\hspace{-2.5mm}{1}} 
\newcommand{\handout}[5]{
   \renewcommand{\thepage}{#1-\arabic{page}}
   \noindent
   \begin{center}
   \framebox{
      \vbox{
    \hbox to 5.78in {{\sf 18.312: Algebraic Combinatorics} 
\hfill \sf #2 }
       \vspace{4mm}
       \hbox to 5.78in { {\Large \hfill #5  \hfill} }
       \vspace{2mm}
       \hbox to 5.78in { {\em #3 \hfill #4} }
      }
   }
   \end{center}
   \vspace*{4mm}
}

\newcommand{\lecture}[4]{\handout{#1}{#2}{Lecture date: #3}{Notes by: #4}{Lecture #1}}


\textwidth=6in
\oddsidemargin=0.25in
\evensidemargin=0.25in
\topmargin=-0.1in
\footskip=0.8in
\parindent=0.0cm
\parskip=0.3cm
\textheight=8.00in
\setcounter{tocdepth} {3}
\setcounter{secnumdepth} {2}
\sloppy

\newtheorem{theorem}{Theorem}
\newtheorem{lemma}[theorem]{Lemma}
\newtheorem{proposition}[theorem]{Proposition}
\newtheorem{corollary}[theorem]{Corollary}
\newtheorem{fact}[theorem]{Fact}
\newtheorem{definition}[theorem]{Definition}
\newtheorem{remark}[theorem]{Remark}
\newtheorem{conjecture}[theorem]{Conjecture}
\newtheorem{question}[theorem]{Question}
\newtheorem{answer}[theorem]{Answer}
\newtheorem{exercise}[theorem]{Exercise}
\newtheorem{example}[theorem]{Example}
\newenvironment{proof}{\noindent \textbf{Proof:}}{$\Box$}

\newcommand{\N}{\mathbb N} % natural numbers 0,1,2,...
\newcommand{\Z}{\mathbb Z}  % integers
\newcommand{\R}{\mathbb R} % reals
\newcommand{\C}{\mathbb C} % complex numbers
\newcommand{\F}{\mathbb F} % finite fields

\newcommand{\floor}[1]{\left\lfloor {#1} \right\rfloor} % floor function
\newcommand{\ceiling}[1]{\left\lceil {#1} \right\rceil} % ceiling function
\newcommand{\binomial}[2]{\left( \begin{array}{c} {#1} \\ 
                        {#2} \end{array} \right)} % binomial coefficients
\newcommand{\modulo}[1]{ (\mbox{mod }{#1})} %congruences
%\newcommand{\modulo}[1]{\quad (\mbox{mod }{#1})} %congruences

\newcommand{\ignore}[1]{} % useful for commenting things out

\makeatother

\usepackage{babel}

\begin{document}
\lecture{1}{Lionel Levine}{February 1, 2011}{Lou Odette}


\section{Names, places, times:}
\begin{itemize}
\item \textbf{Office hours} (in 2-335): \emph{Tuesdays} 12:00-13:00 \& \emph{Wednesdays}
13:00-14:00.
\item \textbf{Course} \textbf{info}: math.mit.edu/\textasciitilde{}levine/18.312
(levine at math dot mit dot edu)
\item \textbf{Grader}: Aldo Pacchiano Camacho (pacchian at mit dot edu)
\item \textbf{Grading}: 

\begin{itemize}
\item weekly PSET worth roughly 50 {}``points'' throughout the term; turn
in your best 30 {}``points'' to count for \emph{30\% of final grade},
\item midterm on \textbf{March 10} worth \emph{20\% of final grade},
\item final on \textbf{May 5} worth \emph{30\% of final grade}, and
\item write-up of notes (1-2 lectures) worth \emph{20\% of final grade}.
\end{itemize}
\end{itemize}

\section{Algebraic Combinatorics}

While the term \emph{combinatorics} frequently is used to refer to
counting problems (Enumerative Combinatorics), some combinatorics
is non-enumerative and in particular Algebraic Combinatorics treats
the relationships between discrete structures and algebraic objects,
e.g.

\begin{center}
{\small }\begin{tabular}{|c|c|c|}
\multicolumn{1}{c}{{\small Discrete Structures}} & \multicolumn{1}{c}{} & \multicolumn{1}{c}{{\small Algebraic Objects}}\tabularnewline
\cline{1-1} \cline{3-3} 
{\small graphs} &  & {\small groups}\tabularnewline
\cline{1-1} \cline{3-3} 
{\small partially ordered sets} & {\small $\dashleftarrow$} & {\small monoids}\tabularnewline
\cline{1-1} \cline{3-3} 
{\small lattices} &  & {\small vector spaces}\tabularnewline
\cline{1-1} \cline{3-3} 
{\small matroids} & {\small $\dashrightarrow$} & {\small rings}\tabularnewline
\cline{1-1} \cline{3-3} 
{\small simplicial complexes} &  & {\small algebras}\tabularnewline
\cline{1-1} \cline{3-3} 
\end{tabular}
\par\end{center}{\small \par}

While the connection with algebraic objects can provide more structure
to the combinatorial objects, the connection can go the other way,
with combinatorial objects providing concrete instances of the algebraic
objects. The next section is an example of a combinatorial object
helping to make an algebraic idea concrete.


\section{Fermat's {}``little'' Theorem}

Fermat's {}``little'' theorem states that if $p$ is a prime number,
then for any $n\in\mathbb{N}$, $n^{p}\lyxmathsym{\textminus}n$ will
be evenly divisible by $p$. To make this concrete, consider the problem
of making a necklace of $p$ beads, choosing the beads from $n$ distinct
colors. Let $\left[n\right]\equiv\{1,2,\ldots,n\}$, and denote the
cardinality of a set $S$ by $\#S$ (alternatively, by $\left|S\right|$).

The set of all necklaces of $p$ beads from $n$ colors has cardinality\[
\#\left\{ f:\left[p\right]\rightarrow\left[n\right]\right\} =n^{p}\]
and so there are $n^{p}-n$ necklaces that aren't all one color (constant).
Let the color of bead $i$ be $f(i)=a_{i}$ and denote a necklace
by a $p$-vector of the colors starting with the bead at 12 o'clock
and proceeding clockwise. For example, the necklace $\underline{a}=(a_{1},a_{2},\ldots,a_{9})$
can also be represented by the following picture

\begin{center}
\includegraphics[scale=0.35]{single_nine_2011_2_4_16_22}
\par\end{center}

Denote a counterclockwise rotation of the necklace by $i$ beads by\[
r^{i}\left(a_{1},a_{2},\ldots.a_{p}\right)=\left(a_{i+1},\ldots.a_{p},a_{1},\ldots,a_{i}\right)\]
so for example $r^{4}\left(a_{1},a_{2},\ldots.a_{9}\right)=\left(a_{5},\ldots.a_{9},a_{1},\ldots,a_{4}\right)$
and the necklaces $\underline{a}$ and $r^{4}\left(\underline{a}\right)$
are illustrated below

\begin{center}
\includegraphics[scale=0.7]{single_nine_rotated_2011_2_4_16_39}
\par\end{center}



\begin{proposition}

If $\underline{a}$ is a non-constant necklace, then the necklaces
$\underline{a},r^{1}\left(\underline{a}\right),\ldots,r^{p-1}\left(\underline{a}\right)$
are all distinct.

\end{proposition}

\begin{proof}

If $r^{i}\left(\underline{a}\right)=r^{j}\left(\underline{a}\right)$
then $\left(a_{i+1},\ldots.a_{p},a_{1},\ldots,a_{i}\right)=\left(a_{j+1},\ldots.a_{p},a_{1},\ldots,a_{j}\right)$
and so for all $k$, $a_{i+k}=a_{j+k}\,\modulo{k}$. Now let $l=j-i$,
then $a_{i+k}=a_{j+k}\Rightarrow a_{k}=a_{k+l}$, and it follows that
$a_{l}=a_{2l}=a_{3l}=\cdots$. However, if $i\ne j$ then $l\ne0$
and $a_{l},a_{2l},a_{3l},\cdots$ are all distinct $\modulo{p}$.
Thus in this case the necklace $\underline{a}$ must be constant,
so $r^{i}\left(\underline{a}\right)=r^{j}\left(\underline{a}\right)$
only if $i=j$. Thus there are exactly $p$ rotational classes on
non constant necklaces, and the number of distinct non-constant necklaces
is evenly divisible by $p$.

\end{proof}


\section{Review of group action.}

Let $G$ be a group and $X$ a non-empty set, with $g,h\in G$ and
$x\in X$.

\begin{definition}

The \emph{action} of $G$ on $X$ is a map\[
G\times X\rightarrow X;\;\left(g,x\right)\mapsto gx\]
such that (i) $\mathbf{1}x=x,\;\forall x\in X$ and (ii) $\forall g,h\in G,\;\forall x\in X$,
$g\left(hx\right)=\left(gh\right)x$. 

\end{definition} 

For each $g\in G$ we also get an invertible map $\sigma(g)$ from
the action of $g$ on $x$ \[
\sigma\left(g\right):X\rightarrow X,\; x\mapsto gx\]
and we can check that $\sigma\left(g\right)$ is invertible \[
\sigma\left(g^{-1}\right)\left(\sigma\left(g\right)x\right)=g^{-1}\left(gx\right)=\left(g^{-1}g\right)x=\mathbf{1}x=x\]
and that the map $\sigma(\cdot)$ is a group homomorphism $\sigma:G\rightarrow\text{Sym}\, X$
(i.e. $\sigma(gh)=\sigma(g)\sigma(h)$) where $\text{Sym}\, X$ is
the symmetric group of all invertible maps. 

\begin{definition}

Let $G$ be a group that acts on the set $X$. If $x\in X$ the \emph{orbit}
of $x$ is the set $\text{Orb}\left(x\right)=\left\{ y\in X|y=gx,\,\text{for some}\, g\in G\right\} $.
The \emph{stabilizer} of $x$ is the set $\text{Stab}\left(x\right)=\left\{ g\in G|gx=x\right\} $.
Notice that $\text{Stab}\left(x\right)$ is a subgroup of $G$ \[
g,h\in\text{Stab}\left(x\right)\Rightarrow gh\left(x\right)=g\left(hx\right)=g\left(x\right)=x\Rightarrow gh\in\text{Stab}\left(x\right)\]


\end{definition}

\begin{definition}

If $S\subset G$, then let $\left\langle S\right\rangle $ denote
the subgroup generated by $S$, the smallest subgroup of $G$ containing
every element of $S$. If $G=\left\langle S\right\rangle $ then $S$
generates $G$ and the elements in $S$ are called \emph{generators}.

\end{definition}

\begin{definition}

A group $G$ is \emph{cyclic} if there exists an element $g\in G$
such that $G=\left\langle g\right\rangle \equiv\left\{ g^{n}|n\in\mathbb{N}\right\} $.

\end{definition}\begin{theorem}

The \emph{Orbit-Stabilizer} Theorem states that $\forall x\in X$\[
\left|\text{Orb}\left(x\right)\right|\times\left|\text{Stab}\left(x\right)\right|=\left|G\right|\]


\end{theorem}


\section{A generalization of Fermat's {}``little'' theorem.}

We can generalize Fermat's {}``little'' theorem if $p$ is not prime
as follows. 
\begin{itemize}
\item In general, $n^{k}-n$ is not evenly divisible by $k$. For example,
consider a necklace of four beads from 2 colors ($n=2,\; k=4$). Clearly\[
\frac{n^{k}-n}{k}=\frac{2^{4}-2}{4}=\frac{14}{4}\notin\mathbb{N}\]
and if we count the non-constant necklaces, the fourteen necklaces
are\[
\underline{a},r^{1}\left(\underline{a}\right),r^{2}\left(\underline{a}\right),r^{3}\left(\underline{a}\right),\underline{b},r^{1}\left(\underline{b}\right),r^{2}\left(\underline{b}\right),r^{3}\left(\underline{b}\right),\underline{c},r^{1}\left(\underline{c}\right),r^{2}\left(\underline{c}\right),r^{3}\left(\underline{c}\right),\underline{d},r^{1}\left(\underline{d}\right)\]
based on the necklaces $\underline{a},\underline{b},\underline{c},\underline{d}$
shown below
\end{itemize}
\begin{center}
\includegraphics{four_two_2011_2_8_17_28}
\par\end{center}
\begin{itemize}
\item For general $k$, consider the set of $k$-necklaces from $n$ colors,\[
N_{k}=\left\{ \left(a_{0},a_{1},\ldots,a_{k-1}\right)|a_{i}\in\left[n\right]\right\} .\]
 $ $Clearly $\left|N_{k}\right|=n^{k}$, and the cyclic group $C_{k}$
of order $k$ acts on members of the set $N_{k}$ by rotation. With
$r^{i}\left(a_{1},a_{2},\ldots.a_{p}\right)=\left(a_{i+1},\ldots.a_{p},a_{1},\ldots,a_{i}\right)$,
taking the indices $\text{mod}\, k$, interpret $r^{i}$ as the group
element acting on the necklace $\underline{a}$. Then for the $k$-necklace
problem $C_{k}=\left\langle r\right\rangle =\left\{ \mathbf{1},r,r^{2},\cdots,r^{k-1}\right\} $
with $r^{k}=\mathbf{1}$ and $X\equiv N_{k}$. Thus\[
r^{i}\left(r^{j}\left(\underline{a}\right)\right)=r^{i+j}\left(\underline{a}\right);\; r^{k}\left(\underline{a}\right)=\underline{a}\]
represents two actions: rotate by $j$, then rotate by $i$. 
\item It is apparent then that the generators of $\underline{a},\underline{b},\underline{c}$
are the cyclic group $C_{1}=\left\langle r\right\rangle $, while
the generator of $\underline{d}$ is the cyclic group $C_{2}=\left\langle r^{2}\right\rangle $.
The two constant necklaces have generator $C_{4}=\left\langle r^{4}\right\rangle $.
\item The necklaces $ $$\underline{a},b,\underline{c}$ each have the same
stabilizer, with $\text{Stab}\left(\underline{a}\right)=\left\langle r^{4}\right\rangle =C_{1}$,
while the stabilizer of the necklace $\underline{d}$ is $\text{Stab}\left(\underline{d}\right)=\left\langle r^{2}\right\rangle =C_{2}$.
The necklaces $\underline{a},b,\underline{c}$ each have the same
orbit, with $\text{Orb}\left(\underline{a}\right)=\left\{ \underline{a},r\underline{a},r^{2}\underline{a},r^{3}\underline{a}\right\} $,
while the orbit of the necklace $\underline{d}$ is $\text{Orb}\left(\underline{d}\right)=\left\{ \underline{d},r^{1}\underline{d}\right\} $.
An application of the Orbit-Stabilizer Theorem gives $\left|\text{Orb}\left(x\right)\right|\times\left|\text{Stab}\left(x\right)\right|=k$
for each necklace.
\end{itemize}
\begin{definition}

A\emph{ }necklace\emph{ }is \emph{primitive} if it has no stabilizer,
i.e. $\text{Stab}\left(\underline{a}\right)=C_{1}$. 

\end{definition}

\begin{example}

Consider the case $k=pq$, where $p,q$ are distinct primes. The goal
is to count the number of \emph{primitive} necklaces. Here $C_{k}=C_{pq}$
which has four subgroups, shown below with their corresponding generators\[
\begin{array}{ccccc}
\text{subgroup:} & C_{pq} & C_{p} & C_{q} & C_{1}\\
 & | & | & | & |\\
\text{generator:} & \left\langle r^{1}\right\rangle  & \left\langle r^{q}\right\rangle  & \left\langle r^{p}\right\rangle  & \left\langle r^{pq}\right\rangle \end{array}\]
Let the set of necklaces $E_{\alpha}=\left\{ \text{necklaces }\underline{a}|\text{Stab}\left(\underline{a}\right)\supseteq C_{\alpha}\right\} $,
with $\alpha\in\left\{ 1,p,q,pq\right\} $. Then the set of primitive
necklaces is the set difference\[
N_{k}-E_{p}-E_{q}\]
and so the number of primitive necklaces is\[
\left|N_{k}\right|-\left|E_{p}\right|-\left|E_{q}\right|+\left|E_{p}\cap E_{q}\right|\]
but since $p$ and $q$ are primes, $E_{p}\cap E_{q}=E_{pq}$, the
number of constant necklaces. So we have\begin{eqnarray*}
\left|N_{k}\right| & = & n^{pq}\\
\left|E_{p}\right| & = & n^{q}\\
\left|E_{q}\right| & = & n^{p}\\
\left|E_{p}\cap E_{q}\right| & = & n\end{eqnarray*}
 and thus the number of \emph{primitive} necklaces is $n^{pq}-n^{q}-n^{p}+n$.

\end{example}
\end{document}
