\documentclass[11pt]{article}
\usepackage{amssymb}
\usepackage{amsfonts}
\usepackage{amsmath}
\usepackage{bm}
\usepackage{latexsym}
\usepackage{epsfig}

\setlength{\evensidemargin}{.25in}
\setlength{\textwidth}{6in}
\setlength{\topmargin}{-0.4in}
\setlength{\textheight}{8.5in}


\newcommand{\handout}[5]{
   \renewcommand{\thepage}{#1-\arabic{page}}
   \noindent
   \begin{center}
   \framebox{
      \vbox{
    \hbox to 5.78in {{\sf 18.312: Algebraic Combinatorics} 
\hfill \sf #2 }
       \vspace{4mm}
       \hbox to 5.78in { {\Large \hfill #5  \hfill} }
       \vspace{2mm}
       \hbox to 5.78in { {\em #3 \hfill #4} }
      }
   }
   \end{center}
   \vspace*{4mm}
}

\newcommand{\lecture}[4]{\handout{#1}{#2}{Lecture date: #3}{Notes by: #4}{Lecture #1}}


\textwidth=6in
\oddsidemargin=0.25in
\evensidemargin=0.25in
\topmargin=-0.1in
\footskip=0.8in
\parindent=0.0cm
\parskip=0.3cm
\textheight=8.00in
\setcounter{tocdepth} {3}
\setcounter{secnumdepth} {2}
\sloppy

\newtheorem{theorem}{Theorem}
\newtheorem{lemma}[theorem]{Lemma}
\newtheorem{proposition}[theorem]{Proposition}
\newtheorem{corollary}[theorem]{Corollary}
\newtheorem{fact}[theorem]{Fact}
\newtheorem{definition}[theorem]{Definition}
\newtheorem{remark}[theorem]{Remark}
\newtheorem{conjecture}[theorem]{Conjecture}
\newtheorem{question}[theorem]{Question}
\newtheorem{answer}[theorem]{Answer}
\newtheorem{exercise}[theorem]{Exercise}
\newtheorem{example}[theorem]{Example}
\newenvironment{proof}{\noindent \textbf{Proof:}}{$\Box$}

\newcommand{\N}{\mathbb N} % natural numbers 0,1,2,...
\newcommand{\Z}{\mathbb Z}  % integers
\newcommand{\R}{\mathbb R} % reals
\newcommand{\C}{\mathbb C} % complex numbers
\newcommand{\F}{\mathbb F} % finite fields

\newcommand{\floor}[1]{\left\lfloor {#1} \right\rfloor} % floor function
\newcommand{\ceiling}[1]{\left\lceil {#1} \right\rceil} % ceiling function
\newcommand{\binomial}[2]{\left( \begin{array}{c} {#1} \\ 
                        {#2} \end{array} \right)} % binomial coefficients
\newcommand{\modulo}[1]{\quad (\mbox{mod }{#1})} %congruences

\newcommand{\ignore}[1]{} % useful for commenting things out



\begin{document}
\lecture{4}{Lionel Levine}{Feb 10, 2011}{Minseon Shin} 
% replace n in the line above (and in the file name) by an actual integer
% replace Feb 1 by the date of the lecture 

\section{Stirling Numbers}

In the previous lecture, the ``signless Stirling number of the first kind'' 
$c(n,k)$ was defined to be the number of permutations $\pi\in S_n$
with exactly $k$ cycles. $c(n,k)$ satisfies the linear recurrence 
$c(n,k) = (n-1)c(n-1,k)+c(n-1,k-1)$.
\begin{lemma}
	\[ \sum_{k=1}^nc(n,k)x^k =  x(x+1)\cdots(x+n-1). \]
\end{lemma}
\begin{proof}
    Induction on $n$. Check that $c(1,1)x=x$. Then
    \begin{align*}
        x(x+1)\cdots(x+n-1) &= \sum_{k=1}^{n-1}c(n-1,k)x^k(x+n-1)\\
        &= \sum_{k=1}^{n-1}c(n-1,k)x^{k+1} + \sum_{k=1}^{n-1}c(n-1,k)(n-1)x^k\\
        &= \sum_{k=1}^{n-1}(c(n-1,k)+(n-1)c(n-1,k))x^{k+1}\\
        &= \sum_{k=1}^n c(n,k)x^k.
    \end{align*}
\end{proof}

\begin{corollary}
\[\#\{\pi\in S_n\;|\;\pi\text{ has an even number of cycles}\} = \#\{\pi\in S_n\;|\;\pi\text{ has an odd number of cycles}\}.\]
\end{corollary}
\begin{proof}
Plugging in $x=-1$ into Lemma 1, we obtain $\sum_{k=1}^n c(n,k)(-1)^k = 0$; 
on the LHS, the sum of terms with positive coefficient is equal to the number of permutations
with an even number of cycles, and the sum of terms with negative coefficient is equal to the 
number of permutations with an odd number of cycles.
\end{proof}

$(-1)^{n-(\#\text{ of cycles in }\pi)} = \text{sgn }\pi$ and the set $\{\pi\in S_n\;|\;\text{sgn }\pi = 1\}$ is called the \textit{alternating group} $A_n$.
By Corollary 2, $|A_n|=n!/2$.

\begin{corollary}
The total number of cycles in all permutations in $S_n$ is equal to $n!\left(\frac{1}{1}+\frac{1}{2}+\ldots+\frac{1}{n}\right)$.
\end{corollary}
\begin{proof}
    The total number of cycles in all permutations in $S_n$ is equal to $\sum_{k=1}^n k\cdot c(n,k)$, which is equal 
    to $C_n'(1)$, where $C_n(x) = \sum_{k=1}^n c(n,k)x^k$. By Lemma 1, $C_n(x) = x(x+1)\cdots(x+n-1)$ so 
    $C_n'(x)$ can also be written $C_n'(x) = C_n(x)\left(\frac{1}{x}+\ldots+\frac{1}{x+n-1}\right)$, which
    evaluated at $x=1$ is $n!\left(\frac{1}{1}+\frac{1}{2}+\ldots+\frac{1}{n}\right)$.
\end{proof}

From Corollary 3 it follows that the average number of cycles in all permutations in $S_n$ is 
$\frac{1}{1}+\frac{1}{2}+\ldots+\frac{1}{n}\approx \ln n$.

\begin{corollary}
    If $p$ is prime, then $c(p,k)$ is divisible by $p$ for $1<k<p$.
\end{corollary}
\begin{proof}
    The polynomial $x(x+1)\cdots(x+n-1)$, considered modulo $p$, has $\{0,1,\ldots,p-1\}$ as roots.
    By Fermat's little theorem, $x^p-x$ has these same roots; therefore, their coefficients must be 
    equal modulo $p$, from which it follows that $[x^k](x(x+1)\cdots(x+n-1)) = [x^k](x^p-x)$
    for all $k$.
\end{proof}

\begin{definition}
    $S(n,k)$, the Stirling number of the second kind, is defined to be the number 
    of partitions of $[n]$ into exactly $k$ nonempty subsets.
\end{definition}

We have $S(n,1)=1$, $S(n,2)=\frac{2^n-2}{2}$, $S(n,n-1)=\binom{n}{2}$, and $S(n,n)=1$.

\begin{lemma}
    $S(n,k)$ satisfies the recurrence
    \[S(n,k) = kS(n-1,k)+S(n-1,k-1).\]
\end{lemma}
\begin{proof}
    Given a partition of $[n-1]$, there are 2 ways to construct a partition 
    of $[n]$ with $k$ subsets: either by adding $n$ into a part of a partition of $[n-1]$ with $k$ subsets
    or by adding the set $\{n\}$ as a new part into a partition of $[n-1]$ with $k-1$ subsets. 
\end{proof}

If $f:[n]\to[k]$ is a surjective function, then the preimages $f^{-1}(1),\ldots,f^{-1}(k)$ partition $[n]$.
There are $k!$ bijective mappings from parts of a partition with $k$ parts to $[k]$. 
Thus $k!S(n,k)$ equals the number of surjective functions $f:[n]\to[k]$.

Using inclusion-exclusion, we find another formula for $S(n,k)$:
let $S$ be the set of all mappings $f:[n]\to [k]$, and $E_i=\{f\;|\;i\not\in\text{im}(f)\}$.
Then $S(n,k) = \left|S-\cup_{i}E_i\right| / k!$. If $I\subset [n]$ with $|I|=r$, then
$\left|\cap_i E_i\right| = (k-r)^n$, so (using the notation in Lecture 2) \[n_r=\sum_{I\subset [n],|I|=r} (k-r)^n = \binom{k}{r}(k-r)^n\]
and \[S(n,k) = \frac{1}{k!}\sum_{i=0}^n\binom{k}{i}(k-i)^n(-1)^i.\]
Convention: for all $k\not\in[n]$ we let $S(n,k) = 0$.

\begin{lemma}
\[\sum_{k=1}^nS(n,k)\cdot x(x-1)\cdots(x-k+1) = x^n.\]
\end{lemma}
\begin{proof}
    It suffices to check the identity for all $x\in \N$.
    \begin{align*}
        x^n &= \#\{\text{all maps }f:[n]\to[x]\}\\
        &= \sum_{k=1}^n \#\{\text{all maps }f:[n]\to[x] \text{ such that }|\text{im}(f)|=k\}\\
        &= \sum_{k=1}^n \binom{x}{k}\#\{\text{surjective maps }f:[n]\to[k]\}\\
        &= \sum_{k=1}^n \binom{x}{k}k!S(n,k)\\
        &= \sum_{k=1}^n x(x-1)\cdots(x-k+1)S(n,k).
    \end{align*}
\end{proof}

\begin{theorem}
    Let $s(n,k) = (-1)^{n-k}c(n,k)$ and $\delta_{mn} = \begin{cases}1&\text{if }m=n\\0&\text{else}\end{cases}$. Then
    \[\sum_{k=n}^{m}S(m,k)s(k,n) = \delta_{mn}.\]
\end{theorem}

\begin{proof} 
    We prove an alternative formulation of Theorem 8.
    Define the $n \times n$ matrices $M=\{s(j,i)\}$ and $N=\{S(j,i)\}$. 
    Since $s(j,i)=S(j,i)=0$ if $j<i$, $M$ and $N$ are upper-triangular.
    Our goal is to prove that 
    \[
    M\times N = 
    \begin{bmatrix}
        s(1,1)&s(2,1)&\cdots&s(n,1)\\
        &s(2,2)&\cdots&s(n,2)\\
        &&\ddots&\vdots\\
        &&&s(n,n)
    \end{bmatrix}\times
    \begin{bmatrix}
        S(1,1)&S(2,1)&\cdots&S(n,1)\\
        &S(2,2)&\cdots&S(n,2)\\
        &&\ddots&\vdots\\
        &&&S(n,n)
    \end{bmatrix}
    =I_n.
    \]
    We prove that $M$ and $N$ are change-of-basis matrices between two particular bases $\boldsymbol{E},\boldsymbol{F}$
    of the vector space $V_n=\{\text{polynomials in }x\text{ of degree at most }n\text{ with constant term }0\}$, where
    \begin{align*}
        \boldsymbol{E} &= (e_1, e_2, \ldots, e_n)\text{ with } e_i = x^i\\
        \boldsymbol{F} &= (f_1, f_2, \ldots, f_n)\text{ with } f_i = x(x-1)\cdots(x-i+1).
    \end{align*}
    In Lemma 1 we substitute $-x$ for $x$ and multiply both sides 
    of the equation by $(-1)^n$ to obtain
    \[f_i = \sum_{k=1}^ie_ks(i,k)\text{ for all }i\quad\implies\quad\boldsymbol{F} = \boldsymbol{E}M\]
    thus $M$ is the change-of-basis matrix from $\boldsymbol{E}$ to $\boldsymbol{F}$.
    By Lemma 7, we have
    \[e_i = \sum_{k=1}^if_kS(i,k)\text{ for all }i\quad\implies\quad\boldsymbol{E} = \boldsymbol{F}N\]
    so $N$ is the change-of-basis matrix from $\boldsymbol{F}$ to $\boldsymbol{E}$. This concludes the proof.
\end{proof}

\section{Linear Recurrences}
Linear operators such as the derivative operator $\frac{d}{dt}$ on the set of differentiable functions $\{f:\R\to\R\}$
have discrete analogues. Let $V$ be the set of sequences $s$ of all real numbers. Then 
the \textit{identity} $I:V\to V$ maps $I(s_0,s_1,s_2,\ldots) = (s_0,s_1,s_2,\ldots)$; 
the \textit{shift operator} $E:V\to V$ maps $E(s_0,s_1,s_2,\ldots) = (s_1,s_2,s_3,\ldots)$; 
the \textit{difference operator} is $D=E-I$ (``discrete derivative'').

The Fibonacci sequence $F(n)$ is defined by $F_1=F_2=1$ and $F_{n+2} = F_{n+1}+F_n$ for $n\ge 1$.
$F_n$ is equal to the number of domino tiliings of a $2\times (n-1)$ rectangle
and also to the number of sequences $(a_1,\ldots,a_{n-2})\in \{0,1\}^{n-2}$ with no two consecutive zeros.
The sequence $F$ satisfies $(E^2-E-I)F= (E-\phi)(E-\overline{\phi})F = 0$ where $\phi = \frac{1+\sqrt{5}}{2}$.
The solutions to $(E-\phi)s=0$ include $\{c\phi^n\}$ and the solutions to $(E-\overline{\phi})s=0$ include $\{c\overline{\phi}^n\}$;
since the linear operators $E$ and $I$ commute, the solutions to $(E^2-E-I)F=0$ include all linear combinations 
$c_1\phi^n+c_2\overline{\phi}^n$. The space of all solutions is a 2-dimensional vector space because there
are 2 degrees of freedom in choosing the first 2 terms of the sequence. Thus $c_1\phi^n+c_2\overline{\phi}^n$
constitute all the solutions to $(E^2-E-I)F=0$.


\end{document}
